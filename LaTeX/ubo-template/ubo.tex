\documentclass{article}
\newcommand{\templatepath}{../../LaTeX/ubo-template/}
\usepackage{\templatepath ubo}
\usepackage[spanish]{babel}
\usepackage{tocloft}
\setlength{\cftbeforesecskip}{0.5em}
\setlength{\cftbeforesubsecskip}{0.2em}
\renewcommand{\cftsecfont}{\bfseries}
\renewcommand{\cftsubsecfont}{\itshape}
\renewcommand{\cftdot}{·}
\renewcommand{\cftsecleader}{\cftdotfill{\cftdotsep}}
\renewcommand{\cftsecpagefont}{\bfseries}
\usepackage{hyperref}
\hypersetup{
	colorlinks=true,
	linkcolor=black,
	urlcolor=blue,
	citecolor=black,
	bookmarks=true,
	bookmarksopen=true,
	bookmarksnumbered=true,
	pdftitle={Limpieza del dataset Spotify 2023 mediante un pipeline},
	pdfauthor={Aldo Hernández}
}

\begin{document}
	% Portada
	\MakeHeader{Análisis de Datos}{Grupo: TEO 1}{Profesor Eliecer Peña Ancavil}{Limpieza del dataset \textit{Spotify 2023} mediante un \textit{pipeline}}{14 de septiembre de 2025}
	\vfill
	\AuthorRowOne{\CustomAuthor{Aldo Hernández}{haldo@pregrado.ubo.cl}{Universidad Autónoma de Nuevo León}{San Nicolás de los Garza, Nuevo León, MX}}
	\newpage
	
	% Índice
	\tableofcontents
	\listoffigures
	\listoftables
	\thispagestyle{empty}
	\newpage
	
	% Contenido
	\section{Resumen}
	
	\section{Introducción}
	\subsection{Descripción del problema}
	\subsection{Objetivos}
	
	\section{Metodología}
	\subsection{Descripción del \textit{dataset}}
	\subsection{Herramientas utilizadas}
	
	\section{Análisis de Datos}
	
	\section{Discusión}
	\subsection{Interpretación de resultados}
	\subsection{Limitaciones}
	\subsection{Posibles mejoras}
	
	\section{Conclusiones}
	
	\section{Bibliografía}
	% Agregar bibliografía
\end{document}